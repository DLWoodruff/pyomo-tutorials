\documentclass[9pt,usenames,dvipsnames]{beamer}

\usetheme[progressbar=frametitle,numbering=fraction]{metropolis}

\title{Methods and Software for Confidence Intervals in Data-based Stochastic Programming}
% \subtitle{A Bootstrap Approach}
\date{}
\author{}
\institute{\inst{1} Graduate School of Management, \\ University of California, Davis}
% \titlegraphic{\hfill\includegraphics[height=1.5cm]{logo.pdf}}
\date{}

\begin{document}

\maketitle

\metroset{sectionpage=none}
\section{Introduction}

\section{Sparse Index Sets \label{sec:sparseindex}}
\label{sec:performance.sparse_sets}

\index{Set@\code{Set}!sparse}
Often, a model makes use of only a portion of the cross product
of multiple index sets. In particular, sometimes the
members in one dimension depend on the value in another, leading
to the slang ``jagged sets.''

For example, a modeler may want to have a constraint to hold for
$$
i \in \mathcal{I},\; k \in \mathcal{K}, \; v \in \mathcal{V}_{k}.
$$
There are many ways to accomplish this, but one good way is to create a
set of tuples composed of all of valid ``model.k, model.V[k]'' pairs.
This is illustrated in the following example where the jagged
set \code{KV} is used.

\autobreaklisting{pyomo/examples/doc/pyomobook/performance-ch/SparseSets.py}{}{1}{23}

An alternative strategy would be to declare the constraint to be indexed by sets
I, K, and V and then use \code{pyo.Constraint.Skip()} to pass over the indices
that are not present. However, in higher dimension, or with large sets this can
cause significant performance degradation.

\end{document}
